\documentclass[11pt]{article}

\usepackage[margin=1in]{geometry}
\usepackage[T1]{fontenc}
\usepackage[utf8]{inputenc}
\usepackage{lmodern}
\usepackage{microtype}
\usepackage{booktabs}
\usepackage{siunitx}
\usepackage{amsmath}
\usepackage{graphicx}
\usepackage{float}
\usepackage{hyperref}
\usepackage{enumitem}

\title{End-to-End EEG-Based Decoding of Visual Oddball Stimulus Class Using Subject-Wise Cross-Validation}
\author{Cole Oliva}
\date{February 20, 2026}

\begin{document}
\maketitle

\begin{abstract}
We built and evaluated an end-to-end EEG analysis pipeline on OpenNeuro dataset ds007056 (PURSUE P300 Visual Oddball). The finalized run processed 285 EEG recordings (286 event files total; 1 EEG file skipped due to read/processing failure) and produced 58,963 trial-level feature rows. Using subject-wise GroupKFold (5 folds) and a class-weighted logistic regression model with subject-normalized feature expansion and fold-wise threshold optimization, the model achieved mean balanced accuracy of 0.6081 (baseline: 0.5000) and ROC AUC of 0.6467. Compared with the pre-upgrade baseline, post-upgrade performance improved by +0.0300 balanced-accuracy points and +0.0385 ROC AUC, with substantial false-positive reduction.
\end{abstract}

\section{Introduction}
\subsection{Research Question}
Can EEG-derived event-locked and spectral features decode target vs non-target stimulus classes above chance under strict subject-wise validation?

\subsection{Hypothesis}
A lightweight EEG feature set (ERP windows + waveform descriptors + bandpower), augmented by subject-wise feature normalization, should provide reliable above-chance decoding under GroupKFold.

\section{Data}
\subsection{Dataset}
\begin{itemize}[leftmargin=*]
  \item Source: OpenNeuro
  \item Accession: ds007056 (v1.1.1)
  \item Task: VisualOddball
  \item Labels: Frequent\_NonTarget, Rare\_Target
  \item Label field: \texttt{value} (mapped via \texttt{task-VisualOddball\_events.json})
\end{itemize}

\subsection{Cohort and Coverage}
\begin{itemize}[leftmargin=*]
  \item Event files processed: 286
  \item EEG files processed: 285
  \item EEG files skipped: 1
  \item Subjects in CV: 285
  \item Trial rows: 58,963
  \item Class counts: Frequent\_NonTarget = 47,171; Rare\_Target = 11,792
\end{itemize}

\section{Methods}
\subsection{Pipeline}
\begin{enumerate}[leftmargin=*]
  \item BIDS indexing and integrity checks
  \item EEG preprocessing (bandpass and baseline handling in feature extraction workflow)
  \item Event-locked feature extraction
  \item Subject-wise cross-validated modeling
  \item Figure and report generation
\end{enumerate}

\subsection{Features}
\begin{itemize}[leftmargin=*]
  \item ERP window means: N1, P2, P3
  \item Waveform features: peak positive, peak negative, peak-to-peak, mean absolute amplitude, standard deviation, absolute AUC
  \item Bandpower features: theta, alpha, beta (+ relative bandpower)
  \item Subject normalization: enabled (z-transformed counterparts for each base feature group)
  \item Feature groups: 15 base + 15 z-normalized
\end{itemize}

\subsection{Modeling}
\begin{itemize}[leftmargin=*]
  \item Model: Logistic Regression (\texttt{class\_weight=balanced})
  \item Validation: GroupKFold (n=5), grouped by subject
  \item Threshold strategy: \texttt{train\_balanced\_optimal} per fold
  \item Primary metric: balanced accuracy
  \item Secondary metric: ROC AUC
  \item Baseline comparator: majority class (balanced accuracy = 0.5)
\end{itemize}

\section{Results}
\subsection{Primary Outcomes}
\begin{itemize}[leftmargin=*]
  \item Mean balanced accuracy (model): 0.6081258625383577
  \item Mean balanced accuracy (baseline): 0.5
  \item Lift over baseline: +0.10812586253835765
  \item ROC AUC: 0.6467288195979253
\end{itemize}

\subsection{Fold Stability}
Fold-wise model balanced accuracy range: 0.598020 to 0.618041.

\subsection{Confusion Matrix (OOF)}
\begin{table}[H]
\centering
\begin{tabular}{lcc}
\toprule
 & Predicted NonTarget & Predicted Target \\
\midrule
True NonTarget & 29,058 & 18,113 \\
True Target & 4,714 & 7,078 \\
\bottomrule
\end{tabular}
\caption{Out-of-fold confusion matrix.}
\end{table}

\subsection{Upgrade Impact}
Relative to pre-upgrade baseline:
\begin{itemize}[leftmargin=*]
  \item Balanced accuracy: +0.0300191153169562
  \item ROC AUC: +0.0384999988638006
  \item TP: +28, FN: -28, FP: -2721, TN: +2721
\end{itemize}

\section{Discussion}
The finalized pipeline demonstrates stable above-chance decoding on a large multi-subject cohort under subject-wise validation. Subject-normalized feature expansion materially improved separability, and threshold optimization provided additional calibration gains. The largest practical gain is the reduction in false positives while maintaining and improving target sensitivity.

\section{Limitations}
\begin{itemize}[leftmargin=*]
  \item One EEG file was skipped, leaving 285 modeled subjects.
  \item Linear model and handcrafted features may underfit nonlinear structure.
  \item Single-dataset evaluation; external generalization is not yet tested.
\end{itemize}

\section{Conclusion}
The project now has a reproducible end-to-end baseline with meaningful decoding performance improvements after feature and threshold upgrades. This constitutes a strong foundation for manuscript-ready iteration and future model and feature expansion.

\section{Reproducibility Artifacts}
\begin{itemize}[leftmargin=*]
  \item Main summary: \texttt{report/results.md}
  \item Preliminary baseline summary: \texttt{report/preliminary\_full\_run\_report.md}
  \item Upgrade deltas: \texttt{report/upgrade\_impact\_report.md}
  \item Final run stamp: \texttt{outputs/metrics/final\_run\_stamp.json}
  \item Model metrics: \texttt{outputs/metrics/modeling\_baseline\_metrics.json}
\end{itemize}

\section{Figures}
\begin{figure}[H]
  \centering
  \includegraphics[width=0.7\textwidth]{class_balance.png}
  \caption{Class balance across extracted trials.}
\end{figure}

\begin{figure}[H]
  \centering
  \includegraphics[width=0.7\textwidth]{fold_balanced_accuracy.png}
  \caption{Fold-wise balanced accuracy (model vs baseline).}
\end{figure}

\begin{figure}[H]
  \centering
  \includegraphics[width=0.6\textwidth]{mean_balanced_accuracy.png}
  \caption{Mean balanced accuracy comparison.}
\end{figure}

\begin{figure}[H]
  \centering
  \includegraphics[width=0.6\textwidth]{confusion_matrix.png}
  \caption{Confusion matrix heatmap.}
\end{figure}

\begin{figure}[H]
  \centering
  \includegraphics[width=0.6\textwidth]{roc_curve.png}
  \caption{ROC curve.}
\end{figure}

\end{document}
